% Arquivo LaTeX de exemplo de dissertação/tese a ser apresentados à CPG do IME-USP
%
% Versão 5: Sex Mar  9 18:05:40 BRT 2012
%
% Criação: Jesús P. Mena-Chalco
% Revisão: Fabio Kon e Paulo Feofiloff
%
% Obs: Leia previamente o texto do arquivo README.txt

\documentclass[$if(fontsize)$$fontsize$,$endif$$if(lang)$$babel-lang$,$endif$$if(papersize)$$papersize$paper,$endif$$for(classoption)$$classoption$$sep$,$endfor$]{$documentclass$}

% ---------------------------------------------------------------------------- %
% Pacotes
\usepackage[T1]{fontenc}
\usepackage[english]{babel}
\usepackage[utf8]{inputenc}
\usepackage[pdftex]{graphicx}           % usamos arquivos pdf/png como figuras
\usepackage{setspace}                   % espaçamento flexível
\usepackage{indentfirst}                % indentação do primeiro parágrafo
\usepackage{makeidx}                    % índice remissivo
\usepackage[nottoc]{tocbibind}          % acrescentamos a bibliografia/indice/conteudo no Table of Contents
\usepackage{courier}                    % usa o Adobe Courier no lugar de Computer Modern Typewriter
\usepackage{type1cm}                    % fontes realmente escaláveis
\usepackage{listings}                   % para formatar código-fonte (ex. em Java)
\usepackage{titletoc}
\usepackage{array}
\usepackage{amssymb,amsmath}
\usepackage{ifxetex,ifluatex}
$if(highlighting-macros)$
$highlighting-macros$
$endif$
\usepackage{hyperref}
$if(colorlinks)$
\PassOptionsToPackage{}{color} % color is loaded by hyperref
$endif$
\hypersetup{unicode=true,
$if(title-meta)$
            pdftitle={$title-meta$},
$endif$
$if(author-meta)$
            pdfauthor={$author-meta$},
$endif$
$if(keywords)$
            pdfkeywords={$for(keywords)$$keywords$$sep$; $endfor$},
$endif$
$if(colorlinks)$
            colorlinks=true,
            linkcolor=$if(linkcolor)$$linkcolor$$else$Maroon$endif$,
            citecolor=$if(citecolor)$$citecolor$$else$Blue$endif$,
            urlcolor=$if(urlcolor)$$urlcolor$$else$Blue$endif$,
$else$
            pdfborder={0 0 0},
$endif$
            breaklinks=true}
\providecommand{\tightlist}{%
  \setlength{\itemsep}{0pt}\setlength{\parskip}{0pt}}
\usepackage{fixltx2e} % provides \textsubscript
%\usepackage[bf,small,compact]{titlesec} % cabeçalhos dos títulos: menores e compactos
\usepackage[fixlanguage]{babelbib}
\usepackage[font=small,format=plain,labelfont=bf,up,textfont=it,up]{caption}
\usepackage[a4paper,headheight=13.6pt,top=2.54cm,bottom=2.0cm,left=2.0cm,right=2.54cm]{geometry} % margens
%\usepackage[pdftex,plainpages=false,pdfpagelabels,pagebackref,colorlinks=true,citecolor=black,linkcolor=black,urlcolor=black,filecolor=black,bookmarksopen=true]{hyperref} % links em preto
% \usepackage[pdftex,plainpages=false,pdfpagelabels,pagebackref,colorlinks=true,citecolor=DarkGreen,linkcolor=NavyBlue,urlcolor=DarkRed,filecolor=green,bookmarksopen=true]{hyperref} % links coloridos
\usepackage[all]{hypcap}                % soluciona o problema com o hyperref e capitulos
\usepackage[square,sort,nonamebreak,comma]{natbib}  % citação bibliográfica alpha (alpha-ime.bst)
\fontsize{60}{62}\usefont{OT1}{cmr}{m}{n}{\selectfont}

% ---------------------------------------------------------------------------- %
% Cabeçalhos similares ao TAOCP de Donald E. Knuth
\usepackage{fancyhdr}
\pagestyle{fancy}
\fancyhf{}
\renewcommand{\chaptermark}[1]{\markboth{\MakeUppercase{#1}}{}}
\renewcommand{\sectionmark}[1]{\markright{\MakeUppercase{#1}}{}}
\renewcommand{\headrulewidth}{0pt}

% ---------------------------------------------------------------------------- %
\frenchspacing                          % arruma o espaço: id est (i.e.) e exempli gratia (e.g.)
\urlstyle{same}                         % URL com o mesmo estilo do texto e não mono-spaced
\makeindex                              % para o índice remissivo
\raggedbottom                           % para não permitir espaços extra no texto
\fontsize{60}{62}\usefont{OT1}{cmr}{m}{n}{\selectfont}
\cleardoublepage
\normalsize

% ---------------------------------------------------------------------------- %
% Opções de listing usados para o código fonte
% Ref: http://en.wikibooks.org/wiki/LaTeX/Packages/Listings
\lstset{ %
language=R,                  % choose the language of the code
basicstyle=\footnotesize,       % the size of the fonts that are used for the code
numbers=left,                   % where to put the line-numbers
numberstyle=\footnotesize,      % the size of the fonts that are used for the line-numbers
stepnumber=1,                   % the step between two line-numbers. If it's 1 each line will be numbered
numbersep=5pt,                  % how far the line-numbers are from the code
showspaces=false,               % show spaces adding particular underscores
showstringspaces=false,         % underline spaces within strings
showtabs=false,                 % show tabs within strings adding particular underscores
frame=single,	                % adds a frame around the code
framerule=0.6pt,
tabsize=2,	                    % sets default tabsize to 2 spaces
captionpos=b,                   % sets the caption-position to bottom
breaklines=true,                % sets automatic line breaking
breakatwhitespace=false,        % sets if automatic breaks should only happen at whitespace
escapeinside={\%*}{*)},         % if you want to add a comment within your code
backgroundcolor=\color[rgb]{1.0,1.0,1.0}, % choose the background color.
rulecolor=\color[rgb]{0.8,0.8,0.8},
extendedchars=true,
xleftmargin=10pt,
xrightmargin=10pt,
framexleftmargin=10pt,
framexrightmargin=10pt
}

% ---------------------------------------------------------------------------- %
% Corpo do texto
\begin{document}
\frontmatter
% cabeçalho para as páginas das seções anteriores ao capítulo 1 (frontmatter)
\fancyhead[RO]{{\footnotesize\rightmark}\hspace{2em}\thepage}
\setcounter{tocdepth}{2}
\fancyhead[LE]{\thepage\hspace{2em}\footnotesize{\leftmark}}
\fancyhead[RE,LO]{}
\fancyhead[RO]{{\footnotesize\rightmark}\hspace{2em}\thepage}

\onehalfspacing  % espaçamento

% ---------------------------------------------------------------------------- %
% CAPA
% Nota: O título para as dissertações/teses do IME-USP devem caber em um
% orifício de 10,7cm de largura x 6,0cm de altura que há na capa fornecida pela SPG.
\thispagestyle{empty}
\begin{center}
    \vspace*{2.3cm}
    \textbf{\Large{$title-formatted$}}\\

    \vspace*{1.2cm}
    \Large{$main-author$}

    \vskip 2cm
    \textsc{
    Dissertação apresentada\\[-0.25cm]
    ao\\[-0.25cm]
    Instituto de Matemática e Estatística\\[-0.25cm]
    da\\[-0.25cm]
    Universidade de São Paulo\\[-0.25cm]
    para\\[-0.25cm]
    obtenção do título\\[-0.25cm]
    de\\[-0.25cm]
    Mestre em Ciências}

    \vskip 1.5cm
    Programa: $masters-program$\\
    Orientador: $advisor$\\

   	\vskip 1cm

$if(financial-support)$
$financial-support$
$endif$

    \vskip 0.5cm
    \normalsize{$date$}
\end{center}

$if(final-submission)$
\newpage
\thispagestyle{empty}
    \begin{center}
        \vspace*{2.3 cm}
	\textbf{\Large{$title-formatted$}}\\
        \vspace*{2 cm}
    \end{center}

    \vskip 2cm

    \begin{flushright}
	Esta versão da dissertação contém as correções e alterações sugeridas\\
	pela Comissão Julgadora durante a defesa da versão original do trabalho,\\
        realizada em $presentation-date$. Uma cópia da versão original está disponível no\\
	Instituto de Matemática e Estatística da Universidade de São Paulo.

    \vskip 2cm

    \end{flushright}
    \vskip 4.2cm

    \begin{quote}
    \noindent Comissão Julgadora:

    \begin{itemize}
		$for(valuation-committee)$
		\item $valuation-committee$
		$endfor$
    \end{itemize}

    \end{quote}
\pagebreak
$else$


% ---------------------------------------------------------------------------- %
% Página de rosto (SÓ PARA A VERSÃO CORRIGIDA - APÓS DEFESA)
% Resolução CoPGr 5890 (20/12/2010)
%
% Nota: O título para as dissertações/teses do IME-USP devem caber em um
% orifício de 10,7cm de largura x 6,0cm de altura que há na capa fornecida pela SPG.
%
% IMPORTANTE:
%   Coloque um '%' em todas as linhas desta
%   página antes de compilar a versão do trabalho que será entregue
%   à Comissão Julgadora antes da defesa
%
%
% ---------------------------------------------------------------------------- %
% Página de rosto (SÓ PARA A VERSÃO DEPOSITADA - ANTES DA DEFESA)
% Resolução CoPGr 5890 (20/12/2010)
%
% IMPORTANTE:
%   Coloque um '%' em todas as linhas
%   desta página antes de compilar a versão
%   final, corrigida, do trabalho
%
%
\newpage
\thispagestyle{empty}
    \begin{center}
        \vspace*{2.3 cm}
        \textbf{\Large{$title-formatted$}}\\
        \vspace*{2 cm}
    \end{center}

    \vskip 2cm

    \begin{flushright}
	Esta é a versão original da dissertação elaborada pelo\\
	candidato $main-author$, tal como \\
	submetida à Comissão Julgadora.
    \end{flushright}

\pagebreak
$endif$


\pagenumbering{roman}     % começamos a numerar

% ---------------------------------------------------------------------------- %
% Agradecimentos:
% Se o candidato não quer fazer agradecimentos, deve simplesmente eliminar esta página
$if(acknowledgements-text)$
$acknowledgements-text$
$endif$


% ---------------------------------------------------------------------------- %
% Resumo
$if(resumo-text)$
$resumo-text$
$endif$

% ---------------------------------------------------------------------------- %
% Abstract

$if(abstract-text)$
$abstract-text$
$endif$

% ---------------------------------------------------------------------------- %
% Sumário
\tableofcontents    % imprime o sumário

% ---------------------------------------------------------------------------- %
$if(abbreviations-text)$
$abbreviations-text$
$endif$

% ---------------------------------------------------------------------------- %
$if(symbols-text)$
$symbols-text$
$endif$

% ---------------------------------------------------------------------------- %
% Listas de figuras e tabelas criadas automaticamente
\listoffigures
\listoftables

% ---------------------------------------------------------------------------- %
% Capítulos do trabalho
\mainmatter

% cabeçalho para as páginas de todos os capítulos
\fancyhead[RE,LO]{\thesection}

\singlespacing              % espaçamento simples
%\onehalfspacing            % espaçamento um e meio

$body$

% ---------------------------------------------------------------------------- %
% Bibliografia
\backmatter \singlespacing   % espaçamento simples
\bibliographystyle{plainnat-ime}% citação bibliográfica alpha
\bibliography{$for(bibliography)$$bibliography$$sep$,$endfor$}

% ---------------------------------------------------------------------------- %

\printindex   % imprime o índice remissivo no documento

\end{document}
